\documentclass[a4paper]{report}
\usepackage[T1]{fontenc}
\usepackage[utf8]{inputenc}
\usepackage{lmodern}
\usepackage{graphicx}
\usepackage[left=3cm,right=3cm,top=3cm,bottom=3cm]{geometry}
\usepackage{eurosym}
\usepackage{fancyhdr}%encabezado y pie de página
\usepackage[colorlinks=true, linkcolor=black, urlcolor=blue]{hyperref}
\setcounter{secnumdepth}{5}
\usepackage[spanish]{babel}
\setcounter{tocdepth}{5}
\usepackage{colortbl}%para colorear tablas
\usepackage{tabularx}
\usepackage{pdfpages}%para incluir documentos pdf
\usepackage{placeins}%para poner barrera y no pasen de secciones los elemntos flotantes
\usepackage{longtable}
\usepackage{multirow} %para juntar varias filas en una tabla

\author{Andoni Martín Reboredo \\ David Ramirez Ambrosi}
\title{\begin{center}
\textbf{\Huge{Práctica 6, Clustering}} \\ \includegraphics{./Figuras/KMeans-density-data.png}\\  \textbf{Minería de datos}
\end{center}}
\date{\today}



\pagestyle{fancy}
\rhead{
\textbf{Minería de datos} \hfill Práctica 6: Clustering
}

\lhead{}

%colores
\definecolor{azul}{RGB}{0,240,255}
\definecolor{amarillo}{RGB}{255,240,0}
\definecolor{rojo}{RGB}{255,198,198}

%Separación entre párrafos
\setlength{\parskip}{4mm}

\begin{document}
\maketitle

\thispagestyle{empty}%para evitar enumeración de la página de la portada y del índice
\newpage
\tableofcontents%índice
\thispagestyle{empty}
\newpage

\listoffigures%índice de figuras
\thispagestyle{empty}
\newpage

\setcounter{page}{1}%Para reinizar el contador de páginas en la página deseada


\chapter{Introducción}

	\section{Clasificación no-supervisada}

	\section{Objetivo}

\chapter{Algoritmo}



\chapter{Diseño}

\chapter{Resultados experimentales}

	\section{Banco de pruebas para la validación de software y resultados}
	
	\section{Resultados}
	
	\section{Clasificación supervisada respecto de otro software de refernecia}
	
	\section{Análisis crítico y discusión de resultados}
	
	\section{Rendimiento del software}

\chapter{Conclusiones}

	\section{Motivación para la realización de clustering \textit{Clustering}}
	
	\section{Conclusiones de los resultados}
	
	\section{Conclusiones generales}
	
	\section{Propuestas de mejora}

%Bibliografía

\newpage
\bibliographystyle{plain}

\bibliography{mdp6}

\chapter{Valoración subjetiva}
	
	\section*{Alcance de objetivos}
	
	\section*{Utilidad de la tarea}
	
	\section*{Dificultad}
	
	\section*{Tiempo de trabajo}
	
	\section*{Sugerencias de mejora}
	
	\section*{Críticas}

\end{document}